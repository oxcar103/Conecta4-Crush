%%
% Plantilla de Memoria
% Modificación de una plantilla de Latex de Nicolas Diaz para adaptarla 
% al castellano y a las necesidades de escribir informática y matemáticas.
%
% Editada por: Mario Román
%
% License:
% CC BY-NC-SA 3.0 (http://creativecommons.org/licenses/by-nc-sa/3.0/)
%%

%%%%%%%%%%%%%%%%%%%%%
% Thin Sectioned Essay
% LaTeX Template
% Version 1.0 (3/8/13)
%
% This template has been downloaded from:
% http://www.LaTeXTemplates.com
%
% Original Author:
% Nicolas Diaz (nsdiaz@uc.cl) with extensive modifications by:
% Vel (vel@latextemplates.com)
%
% License:
% CC BY-NC-SA 3.0 (http://creativecommons.org/licenses/by-nc-sa/3.0/)
%
%%%%%%%%%%%%%%%%%%%%%

%----------------------------------------------------------------------------------------
%	PAQUETES Y CONFIGURACIÓN DEL DOCUMENTO
%----------------------------------------------------------------------------------------

%% Configuración del papel.
% microtype: Tipografía.
% mathpazo: Usa la fuente Palatino.
\documentclass[a4paper, 11pt]{article}
\usepackage[protrusion=true,expansion=true]{microtype}
\usepackage{mathpazo}

% Indentación de párrafos para Palatino
\setlength{\parindent}{0pt}
  \parskip=8pt
\linespread{1.05} % Change line spacing here, Palatino benefits from a slight increase by default


%% Castellano.
% noquoting: Permite uso de comillas no españolas.
% lcroman: Permite la enumeración con numerales romanos en minúscula.
% fontenc: Usa la fuente completa para que pueda copiarse correctamente del pdf.
\usepackage[spanish,es-noquoting,es-lcroman]{babel}
\usepackage[utf8]{inputenc}
\usepackage[T1]{fontenc}
\selectlanguage{spanish}


%% Matemáticas
\usepackage{amsmath}


%% Tablas
\usepackage{multirow} 

%Referencias
\usepackage[hidelinks]{hyperref}

\makeatletter

%----------------------------------------------------------------------------------------
%	TÍTULO
%----------------------------------------------------------------------------------------
% Configuraciones para el título.
% El título no debe editarse aquí.
\renewcommand{\maketitle}{
  \begin{flushright} % Right align
  
  {\LARGE\@title} % Increase the font size of the title
  
  \vspace{50pt} % Some vertical space between the title and author name
  
  {\large\@author} % Author name
  \\\@date % Date
  \vspace{40pt} % Some vertical space between the author block and abstract
  \end{flushright}
}

% Título
\title{\textbf{Inteligencia Artificial}\\ % Title
Métodos de búsqueda con adversarios: Conecta-4 Crush} % Subtitle

\author{\textsc{Óscar Bermúdez Garrido} % Author
\\{\textit{Universidad de Granada}}} % Institution

\date{\today} % Date



%----------------------------------------------------------------------------------------
%	DOCUMENTO
%----------------------------------------------------------------------------------------

\begin{document}

\maketitle % Print the title section
% Índice
{\parskip=2pt
  \tableofcontents
}
\pagebreak

%% Inicio del documento

\section{Análisis del problema}
	El problema consiste en desarrollar una estrategia del juego Conecta-4 Crush, un juego similar
	al clásico Conecta-4 pero con una ligera variante debida a la introducción de una ficha especial,
	la $"$ficha bomba$"$.
	
	Esta ficha provoca una distorsión en su entorno si decide utilizarse su funcionalidad, lo que
	provoca un cambio de mentalidad a la hora de la búsqueda de la estrategia para poder obtener
	la victoria en este juego respecto a su predecesor.

\section{Descripción de la solución planteada}
	\subsection{Introducción}
		El jugador de Conecta-4 Crush que he implementado para tomar sus decisiones se basa en
		el \href{https://en.wikipedia.org/wiki/Alpha\%E2\%80\%93beta\_pruning}{algoritmo de poda
		$\alpha$-$\beta$} y en una \hyperref[sec: h(x)]{función heurística de invención propia}.
		
	\subsection{Algoritmo de poda $\alpha$-$\beta$}
		
		A grandes rasgos, el algoritmo de poda $\alpha$-$\beta$ consiste en que, dado un nodo:
		\begin{itemize}
			\item Si es terminal o se ha llegado al máximo de profundidad impuesto(en caso de
			existir; para pequeños problemas, no existe este límite), se evalúa la función
			heurística en el nodo.
			
			\item Si no es terminal, se realiza la ejecución recursiva del algoritmo a cada uno
			de sus hijos uno a uno hasta que el máximo(suponiendo que el nodo es MAX, si es
			MIN, sería el mínimo) de los valores devueltos por sus hijos supere la cota
			superior(respectivamente, inferior) o se hayan evaluado todos.
		\end{itemize}
		
		Para la implementación de dicho algoritmo, nos hemos restringido a una profundidad en el
		árbol de desarrollo del juego hasta profundidad 8.
		
	\subsection{Heurística}\label{sec: h(x)}
		La heurística empleada como valoración de los nodos terminales consiste en que:
		\begin{itemize}
			\item Si nos encontramos en un tablero final, (es decir, con ganador), se tiene que:
			\begin{itemize}
				\item Si somos nosotros el ganador, se otorga la valoración máxima posible.
				\item Si nuestro adversario es el ganador, se otorga la valoración mínima posible.
			\end{itemize}
			
			\item Si nos encontramos en un tablero no final sin casillas libres, quiere decir que
			es un empate, se otorga el valor neutro 0.
			
			\item En otro caso:
				\begin{itemize}
					\item Para cada ficha disponible en el tablero, buscamos qué valor tiene dicha
					ficha mediante las funciones de comprobación y conteo que se explicarán más
					adelante.
					
					\item Según la ficha que estemos explorando sea nuestra o enemiga, se procede
					respectivamente a sumar o restar el valor que le corresponde a la ficha.
					
					\item El valor del tablero es la suma de todos los valores de las fichas
					dispuestas en él que es el que será otorgado.
				\end{itemize}
										
		\end{itemize}

		Finalmente, se devuelve el valor otorgado, el cuál será utilizado de forma conveniente
		por el algoritmo de poda $\alpha$-$\beta$ antes descrito.
	
		Para la realización de la heurística, se ha realizado la implementación de una serie de
		funciones descritas a continuación:

		\subsubsection{Funciones de comprobación y conteo}
			Para la comprobación de las líneas en línea recta hemos subdividido el problema en 3
			subproblemas más fáciles de implementar tanto por la sencillez de las funciones como
			por la estructuración de los mismos.
			
			Los subproblemas serían líneas horizontales, verticales y diagonales.
			
			A \textit{grosso modo}, en cada subproblema:
			\begin{itemize}
				\item En cada una de las casillas desde 3 posiciones hacia atrás(para asegurarnos
				que la ficha pertenece) hasta la posición de la ficha, buscamos en ésa y las 3
				siguientes, dando que no hay posibilidad de poner una línea si:
				\begin{itemize}
					\item Alguna casilla está fuera del mapa.
					\item Alguna casilla está ocupada por una ficha del adversario.
				\end{itemize}
				
				\item Si hay posibilidad de poner una línea, contamos en cuántas casillas tenemos
				ya puesta alguna ficha.
				
				\item En función del número de fichas encontradas en el paso anterior y de la
				ponderación establecida(que se verá en el siguiente apartado), se añade un valor
				a la puntuación que actualmente llevamos de dicho subproblema.
				
				\item Finalmente, se devuelve el valor del subproblema
			\end{itemize}
			
			Para acabar, sumamos los valores obtenido por cada subproblema, obteniendo así el valor
			de la casilla que nos pidieron, y lo devolvemos.
	
		\subsubsection{Ponderaciones}
			Hay dos tipos de ponderaciones: Valores extremos y valoraciones del número de fichas.
			
			Los valores extremos son valores que sólo se pueden alcanzar si se asignan por haber
			perdido o ganado, estos valores son -9999999999.0 y 9999999999.0, respectivamente.
		
			Los valores del número de fichas se adjudican por la importancia que queremos darle
			a que salgan un número de fichas en una misma línea para obtener la victoria.
		
			Básicamente, los valores dados a cada una de las veces que salían 1, 2 ó 3 puntos
			que pudiesen pertenecer a una misma línea se basa en el ensayo-error buscando obtener
			unos valores que evitasen perder fácilmente, por ejemplo, obstruyendo una línea
			horizontal de 3 del enemigo.
			
			Se han realizado pruebas con las siguientes tuplas de valores:
			\begin{itemize}
				\item (1,2,3)
				\item (1, 5, 10)
				\item (1, 10, 100)
				\item (1, 20, 50)
				\item (1, 20, 100)
				\item (1, 25, 100)
				\item (1, 25, 1000)
				\item (1, 50, 1000)
				\item (1, 50, 100000)
				\item (1, 100, 1000000)
			\end{itemize}
			
			Finalmente, nos hemos quedado con (1, 50, 1000) porque fue el que mejores resultados dio.
\end{document}
